\documentclass[../informe_krapp.tex]{subfiles}
\begin{document}
\graphicspath{{../images/}}
{
\renewcommand{\subsectionbreak}{}
\section{Introducción}

Librepass es un sistema FOSS(Free and Open Source) de seguridad para empresas.
Al ser FOSS, está hosteado en
\href{https://github.com/KrappRamiro/librepass}{un repositorio público en GitHub} ---
{\small \url{https://github.com/KrappRamiro/librepass}}.

Fue desarrollado usando una placa de desarrollo DOIT ESP32 DevKit V1, conectado
a un array de lectores RFID-RC552.

Estos lectores son capaces de leer un sistema de tarjetas y/o llaveros RFID con un
código hexadecimal indentificador, el cual se asigna a cada empleado de la empresa, y
sirve para identificar al empleado.

A nivel de hardware, hay 3 componentes involucrados:
\begin{enumerate}
	\item El microcontrolador: un ESP32
	\item EL PCD (Proximity Coupling Device): RFID-MFRC522
	\item El PICC (Proximity Integrated Circuit Card): Una tarjeta o llavero usando
	      la interfaz ISO 14443A
\end{enumerate}

\subsection{El proyecto y el Software Libre}
En el desarrollo de este proyecto, se planteó usar la filosofía del software libre.
Segun GNU \cite{gnu}:
\begin{center}
	\rule{0.8\textwidth}{0.3pt}
\end{center}
``«Software libre» es el software que respeta la libertad de los usuarios y la comunidad. A grandes rasgos, significa que los usuarios tienen la libertad de ejecutar, copiar, distribuir, estudiar, modificar y mejorar el software. Es decir, el «software libre» es una cuestión de libertad, no de precio. Para entender el concepto, piense en «libre» como en «libre expresión», no como en «barra libre». En inglés, a veces en lugar de «free software» decimos «libre software», empleando ese adjetivo francés o español, derivado de «libertad», para mostrar que no queremos decir que el software es gratuito.

Puede haber pagado dinero para obtener copias de un programa libre, o puede haber obtenido copias sin costo. Pero con independencia de cómo obtuvo sus copias, siempre tiene la libertad de copiar y modificar el software, incluso de vender copias.

(...)

Un programa es software libre si los usuarios tienen las cuatro libertades esenciales:
\begin{itemize}
	\item La libertad de ejecutar el programa como se desee, con cualquier propósito (libertad 0).
	\item La libertad de estudiar cómo funciona el programa, y cambiarlo para que haga lo que se desee (libertad 1). El acceso al código fuente es una condición necesaria para ello.
	\item La libertad de redistribuir copias para ayudar a otros (libertad 2).
	\item La libertad de distribuir copias de sus versiones modificadas a terceros (libertad 3). Esto le permite ofrecer a toda la comunidad la oportunidad de beneficiarse de las modificaciones. El acceso al código fuente es una condición necesaria para ello.
\end{itemize}

Un programa es software libre si otorga a los usuarios todas estas libertades de manera adecuada. De lo contrario no es libre. Existen diversos esquemas de distribución que no son libres, y si bien podemos distinguirlos en base a cuánto les falta para llegar a ser libres, nosotros los consideramos contrarios a la ética a todos por igual.''
\begin{center}
	\rule{0.8\textwidth}{0.3pt}
\end{center}

\clearpage

\subsection{Licencia}
Se escogió usar la licencia MIT \cite{mit}, la cual, en ingles, es la siguiente:

Copyright (c) 2022 Krapp Ramiro

Permission is hereby granted, free of charge, to any person obtaining a copy
of this software and associated documentation files (the "Software"), to deal
in the Software without restriction, including without limitation the rights
to use, copy, modify, merge, publish, distribute, sublicense, and/or sell
copies of the Software, and to permit persons to whom the Software is
furnished to do so, subject to the following conditions:

The above copyright notice and this permission notice shall be included in all
copies or substantial portions of the Software.

THE SOFTWARE IS PROVIDED ``AS IS'', WITHOUT WARRANTY OF ANY KIND, EXPRESS OR
IMPLIED, INCLUDING BUT NOT LIMITED TO THE WARRANTIES OF MERCHANTABILITY,
FITNESS FOR A PARTICULAR PURPOSE AND NONINFRINGEMENT. IN NO EVENT SHALL THE
AUTHORS OR COPYRIGHT HOLDERS BE LIABLE FOR ANY CLAIM, DAMAGES OR OTHER
LIABILITY, WHETHER IN AN ACTION OF CONTRACT, TORT OR OTHERWISE, ARISING FROM,
OUT OF OR IN CONNECTION WITH THE SOFTWARE OR THE USE OR OTHER DEALINGS IN THE
SOFTWARE.


}

\end{document}