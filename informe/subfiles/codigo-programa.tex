\documentclass[../informe_krapp.tex]{subfiles}
\usemintedstyle{vim}

\begin{document}
\section{Codigo del programa}
El codigo de programa fue escrito usando Visual Studio Code, usando la extensión
de platformIO con el framework de Arduino.

Para el versionado del código, se uso Git \url{https://git-scm.com/}, un programa
FOSS estandar en la industria.
Para una mejor organización, se dividió en tres branches principales:
\begin{enumerate}
	\item Una branch main, con las versiones estables del código.
	\item Una branch dev, con las versiones de desarrollo.
	\item Una branch dev-tema-a-desarrollar, (reemplazando tema-a-desarrollar por el
	      tema que se desarrolla, se usaba una de estas y despues se mergeaba con dev)
\end{enumerate}


\inputminted
[frame= lines, linenos, breaklines, tabsize = 3, fontsize=\footnotesize,
	label=Archivo platformio.ini usado para la configuracion de platformIO]
{ini}{../platformio.ini}

% -------------- Codigo del ESP32 ---------------

\inputminted %TODO: ponerle una mejor font
[frame= lines, linenos, breaklines, tabsize = 3, fontsize=\footnotesize,
	label=Codigo del ESP32]
{cpp}{../src/main.cpp}

\inputminted
[frame= lines, linenos, breaklines, tabsize = 3, fontsize=\footnotesize,
	label=Archivo de utilidades]
{cpp}{../src/krapp_utils.cpp}

\inputminted
[frame= lines, linenos, breaklines, tabsize = 3, fontsize=\footnotesize,
	label=Imagenes del OLED]
{cpp}{../include/oled_images.h}

% ---------------- WEBAPP ------------------

\inputminted
[frame= lines, linenos, breaklines, tabsize = 3, fontsize=\footnotesize,
label=Archivo init para la creacion del paquete webapp]
{py}{../webapp/__init__.py}

\inputminted
[frame= lines, linenos, breaklines, tabsize = 3, fontsize=\footnotesize,
	label=Modelos de la base de datos]
{py}{../webapp/db_models.py}

\inputminted
[frame= lines, linenos, breaklines, tabsize = 3, fontsize=\footnotesize,
	label=Forms usados para el ingreso de datos]
{py}{../webapp/forms.py}

\inputminted
[frame= lines, linenos, breaklines, tabsize = 3, fontsize=\footnotesize,
	label=Codigo base del html]
{html}{../webapp/templates/base.html}

\inputminted
[frame= lines, linenos, breaklines, tabsize = 3, fontsize=\footnotesize,
	label=Codigo base del html]
{html}{../webapp/templates/add_door.html}

\inputminted
[frame= lines, linenos, breaklines, tabsize = 3, fontsize=\footnotesize,
	label=Pagina para añadir empleados]
{html}{../webapp/templates/add_employee.html}

\inputminted
[frame= lines, linenos, breaklines, tabsize = 3, fontsize=\footnotesize,
	label=Pagina para borrar puertas]
{html}{../webapp/templates/delete_door.html}

\inputminted
[frame= lines, linenos, breaklines, tabsize = 3, fontsize=\footnotesize,
	label=Pagina para borrar empleados]
{html}{../webapp/templates/delete_employee.html}

\inputminted
[frame= lines, linenos, breaklines, tabsize = 3, fontsize=\footnotesize,
	label=Pagina para editar puertas]
{html}{../webapp/templates/edit_door.html}

\inputminted
[frame= lines, linenos, breaklines, tabsize = 3, fontsize=\footnotesize,
	label=Pagina para editar empleados html]
{html}{../webapp/templates/edit_employee.html}

\inputminted
[frame= lines, linenos, breaklines, tabsize = 3, fontsize=\footnotesize,
	label=Pagina de inicio html]
{html}{../webapp/templates/index.html}

\inputminted
[frame= lines, linenos, breaklines, tabsize = 3, fontsize=\footnotesize,
	label=Pagina para iniciar sesion]
{html}{../webapp/templates/login.html}

\inputminted
[frame= lines, linenos, breaklines, tabsize = 3, fontsize=\footnotesize,
	label=Pagina para registrarse]
{html}{../webapp/templates/register.html}

\inputminted
[frame= lines, linenos, breaklines, tabsize = 3, fontsize=\footnotesize,
	label=Pagina para ver los accesos]
{html}{../webapp/templates/see_accesses.html}

\inputminted
[frame= lines, linenos, breaklines, tabsize = 3, fontsize=\footnotesize,
	label=Pagina para ver las puertas]
{html}{../webapp/templates/see_doors.html}

\inputminted
[frame= lines, linenos, breaklines, tabsize = 3, fontsize=\footnotesize,
	label=Pagina para ver los empleados]
{html}{../webapp/templates/see_employees.html}

\end{document}