\documentclass[../informe_krapp.tex]{subfiles}
\begin{document}

\section{Bitacoras Personales}
\subsection{Krapp Ramiro}
\subsubsection{24/03/2022}
\begin{itemize}
	\item Comence creando un repositorio en github para subir todos los cambios del proyecto
	\item Cree un codigo en C++, para definir un sistema de clases.
	      La idea es hacer una clase Tren, para que sirva de blueprint para todos los trenes,
	      y una clase Persona, para que sea padre de otras dos clases, Maquinista y Pasajero.
	      Al pasajero le voy a asignar una sube, y al maquinista le voy a asignar
	      un salario y un seniority.
\end{itemize}
\subsubsection{25/03/2022}
\todo{Hacer las urls mas chicas con \small o tiny}
\begin{itemize}
	\item Pienso implementar la sube con un sistema usando RFID\\
	      \small{\url{https://randomnerdtutorials.com/security-access-using-mfrc522-rfid-reader-with-arduino/}}
	\item La idea seria armar un sistema en el que cada usuario pueda tener un llavero
	      RFID, y que asigne ese llavero RFID con una cuenta.
	      Tambien necesito comprar los lectores para RFID.
	      En total, tengo pensado comprar 2 lectores y 4 llaveros RFID.
	      Por qué 2 lectores? Estaba pensando en asignar cada uno a una estación distinta.
	      Por qué 4 llaveros? Estaba pensando en asignar cada uno a un pasajero distinto.
	\item Encontre que para en \LaTeX \ dejar de tener problema con las url yendose fuera
	      pantalla, puedo usar el paquete url con la opcion [hyphens], lo
	      unico es que hay que cargar este paquete antes de hyperref.
	      Esto es porque por defecto el paquete hyperref ya carga al paquete url
	      \url{https://tex.stackexchange.com/questions/544671/option-clash-for-package-url-urlstyle}

\end{itemize}
\subsubsection{26/03/2022}
\begin{itemize}
	\item Encontre mucha documentacion del ESP32 y de proyectos con el RFID, la principal es esta:
	\item \url{https://arduinogetstarted.com/tutorials/arduino-rfid-nfc}
	\item \url{https://olddocs.zerynth.com/latest/official/board.zerynth.doit_esp32/docs/index.html}
	\item \url{https://testzdoc.zerynth.com/reference/boards/doit_esp32/docs/}
	\item \url{https://randomnerdtutorials.com/esp32-pinout-reference-gpios/}
	\item \url{https://randomnerdtutorials.com/getting-started-with-esp32/}
	\item Voy a usar el grafico de randomnerdtutorials, del link de getting-started...,
	      el que incluye que pines son GPIO, me va a servir un montón.
	      Para cuando quiera programar, solamente tengo que recordar que lo mejor es usar los
	      GPIO del 13 al 33, y que mi DOIT ESP32 DevKit V1 es la version de 30 pines
	\item Decidi seguir el tutorial de este link
	      \url{https://www.instructables.com/ESP32-With-RFID-Access-Control/}
	\item Hice andar el codigo durante un tiempo, grabe que funcionaba incluso, pero de
	      repente dejo de funcionar, solamente me da un error:
	      PCD\_Authenticate() failed: Timeout in communication.
	\item Creo que se por qué dejó de funcionar, me parece que cortocircuité algo
	      con el la parte de metal del llavero, me parece haber cortocircuitado
	      los pines del sensor RFID-RC552
\end{itemize}

\subsubsection{27/03/2022}
\begin{itemize}
	\item Hice una branch nueva en git para trabajar exclusivamente en el informe,
	      la llamé update\_informe.
	      Aproveché para eliminar la sección Base de Datos, que me había quedado ahí
	      de un copypaste de un proyecto anterior.
\end{itemize}

\subsubsection{28/03/2022}
\begin{itemize}
	\item Cometi un error haciendo un stash en git y elimine parte del trabajo que hice
	      en el informe :'(
	\item Encontre este codigo que me puede servir

	      \url{https://esp32io.com/tutorials/esp32-rfid-nfc}
	\item Tambien encontre la documentacion de la libreria para los RFID que usa el
	      protocolo de comunicacion MFRC, pero como usa SPI no se como meter varios RFID en
	      paralelo sin usar RFID, lo tendria que investigar
	      \url{https://www.arduino.cc/reference/en/libraries/mfrc522/}
\end{itemize}

\subsubsection{29/03/2022}
\begin{itemize}
	\item Investigando info para hacer el informe y saber más sobre SPI, encontre esto:
	      \begin{itemize}
		      \item \url{https://www.arduino.cc/en/reference/SPI}
		      \item \url{https://arduinogetstarted.com/faq/how-to-connect-multiple-spi-sensors-devices-with-arduino} (Especialmente útil para conectar multiples dispositivos)
		      \item \url{https://arduinogetstarted.com/tutorials/arduino-rfid-nfc}
		      \item \url{https://www.exostivlabs.com/files/documents/Introduction-to-I2C-and-SPI-Protocols.pdf?/article/aa-00255/22/introduction-to-spi-and-ic-protocols.html}
		      \item \url{https://www.corelis.com/education/tutorials/spi-tutorial/}
		      \item \url{https://learn.sparkfun.com/tutorials/serial-peripheral-interface-spi/}
	      \end{itemize}
	\item Creo que para lo que quiero hacer me sirve la conexion daisy-chained
	      del protocolo SPI
	\item Para las bibliografias, voy a usar esto \url{https://latex-tutorial.com/tutorials/bibtex/}
	\item tambien este tutorial sirve \url{https://www.overleaf.com/learn/latex/Bibliography_management_with_biblatex}
	\item Estuve trabajando en el informe, hice gran parte de la sección del SPI

	      \subsubsection{30/03/2022}
	      \begin{itemize}
		      \item Hoy estoy trabajando en la documentacion del sistema RFID, mientras espero
		            que lleguen los componentes que compré. Estoy haciendo la documentación porque
		            me sirve para estudiar y ya entrar a armar cosas con más conocimiento.
	      \end{itemize}
\end{itemize}
\end{document}



