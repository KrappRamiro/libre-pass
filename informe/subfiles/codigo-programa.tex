\documentclass[../informe_krapp.tex]{subfiles}
\begin{document}
\section{Codigo del programa}
El codigo de programa fue escrito usando Visual Studio Code, usando la extensión
de platformIO con el framework de Arduino.

Para el versionado del código, se uso Git \url{https://git-scm.com/}, un programa
FOSS estandar en la industria.
Para una mejor organización, se dividió en tres branches principales:
\begin{enumerate}
	\item Una branch main, con las versiones estables del código.
	\item Una branch dev, con las versiones de desarrollo.
	\item Una branch dev-tema-a-desarrollar, (reemplazando tema-a-desarrollar por el
	      tema que se desarrolla, se usaba una de estas y despues se mergeaba con dev)
\end{enumerate}

\inputminted %TODO: ponerle inconsolata
[frame= lines, linenos, breaklines, tabsize = 3, fontsize=\footnotesize,
	label=Codigo principal]
{cpp}{../src/main.cpp}


\end{document}